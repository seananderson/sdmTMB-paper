\documentclass[letterpaper]{letter}
\usepackage[margin=1.50in]{geometry}
\geometry{letterpaper}
\usepackage{graphicx}
\addtolength{\topmargin}{-30pt}
\addtolength{\textheight}{55pt}

\usepackage{textcomp}
% \usepackage[sb]{libertine}
% \usepackage[varqu,varl]{inconsolata}% sans serif typewriter
% \usepackage[libertine,bigdelims,vvarbb]{newtxmath} % bb from STIX
% \usepackage[cal=boondoxo]{mathalfa} % mathcal
% \useosf % osf for text, not math
% \usepackage[supstfm=libertinesups,%
%   supscaled=1.2,%
%   raised=-.13em]{superiors}

\providecommand{\pkg}[1]{{\normalfont\fontseries{b}\selectfont #1}}
\let\proglang=\textsf

\usepackage[dvipsnames]{xcolor}
\definecolor{niceblue}{HTML}{236899}
\newcommand{\rev}[1]{{\color{niceblue} #1}}
\usepackage{hyperref}
\hypersetup{
    colorlinks = true,
    linkcolor = black,
    filecolor = black,
    urlcolor = blue,
    citecolor = blue,
    linkbordercolor = white
}

% \signature{Sean Anderson}
\address{
  % Sean C. Anderson\\
  % Research Scientist \\
  % Fisheries and Oceans Canada \\
  % Pacific Biological Station \\
  % Nanaimo, BC, Canada \\
  % sean.anderson@dfo-mpo.gc.ca
}

\begin{document}
\begin{letter}{}
\pagestyle{empty}

\opening{Dear Editorial Board,}

Thank you for considering our manuscript for publication in the Journal of Statistical Software. Our manuscript introduces the \proglang{R} package \pkg{sdmTMB} for fitting a broad class of generalized linear mixed effect models (GLMMs) that include spatial or spatiotemporal random fields. Such models are useful across a wide array of disciplines.

We believe that \pkg{sdmTMB} represents a major advance in terms of speed, functionality, and accessibility for fitting SPDE (stochastic partial differential equation)-based GLMMs to spatial or spatiotemporal data.

Speed: The package combines the SPDE approximation with estimation in the \pkg{TMB} \proglang{R} package for computational efficiency. Across our comparisons, \pkg{sdmTMB} was 6--24 times faster than the \pkg{INLA} or \pkg{inlabru} packages, the next fastest related software.

Functionality: This package represents a spatial analog of the \pkg{glmmTMB} \proglang{R} package, with additional features beyond random fields (penalized smoothers as in \pkg{mgcv}, priors, threshold/breakpoint models) and spatial anisotropy (directionally dependent spatial correlation; lacking in \pkg{INLA} and \pkg{inlabru}). Spatial anisotropy is critical in many applied contexts such as when modelling species distributions near any geographic boundary or along an environmental gradient.

Accessibility: \pkg{sdmTMB} is a much more usable interface than alternatives for applied users, who may be familiar with \proglang{R}'s \texttt{glm()}, \pkg{glmmTMB}, \pkg{lme4}, or \pkg{mgcv}. In our experience, such applied users struggle to code these types of models in \pkg{INLA}; \pkg{inlabru} improves the experience but still requires an understanding of \pkg{INLA} to fit many useful but complex structures. Furthermore, \pkg{INLA} and \pkg{inlabru} are fundamentally Bayesian and require users to deal with the complexity of priors. \pkg{sdmTMB} can incorporate priors and extend to full Bayesian inference via \pkg{Stan}, while by default allowing users the simplicity and speed of maximum likelihood.

As evidence of its utility, there are already \href{https://github.com/pbs-assess/sdmTMB/wiki/Publications-using-sdmTMB}{16 published preprints, papers, or reports} using \pkg{sdmTMB}, six funded research grants, an ever-growing number of ongoing research projects, and workshops that in total have been attended by $>$200 participants.

% We think the readers of JSS would be the ideal audience for our manuscf

JSS would be the ideal outlet to provide a citable resource for \pkg{sdmTMB} and to introduce the package to a wider audience. \pkg{sdmTMB} is an open-source \proglang{R} package \href{https://CRAN.R-project.org/package=sdmTMB}{published on CRAN}. Our code is fully reproducible and included with our submission. We have posted a preprint version of our manuscript \href{https://doi.org/10.1101/2022.03.24.485545}{on bioRxiv}. We look forward to constructive feedback from your reviewers and editors.

\vspace{3mm}
\closing{Sincerely,\\
\fromsig{\includegraphics[scale=0.42]{/Users/seananderson/Dropbox/sig.pdf}}
}

\end{letter}
\end{document}
